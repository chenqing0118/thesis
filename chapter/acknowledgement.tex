\begin{acknowledgement}
    
    {
    在南京大学的两年研究生生涯如白驹过隙,很快又要踏上新的人生旅途。这段时间里一个不得不提的话题便是新冠疫情,
    疫情初期是我准备研究生考试的日子,在我即将毕业时,新冠疫情似乎终于成为一段历史。疫情给我的学习生涯带来了
    些许不便,但同样也赋予了这一段时光特别的含义。时光飞逝但不代表回忆寡淡,在此我想感谢每一位为生活带来温暖的人。
    \par
    首先要感谢我的父母。自从开启住校的学习生活,我与父母便是聚少离多,进入研究生阶段更是如此。但虽然相距千里,每一通电话里都能感受到就在背后的支持。父母对我的信任也是这两年我能够自信面对各种人生第一次的重要原由,真正觉得自己可以独立面对社会。
    疫情管制全面放开后,最年轻力壮的我竟然成为家里症状最重的患者,在难熬的一个月里我又觉得自己始终是他们的孩子,有父母照顾的感觉真幸福。父母给了我一个幸福的原生家庭,我也一定要让这份幸福延续下去。
    \par
    其次感谢同在南京租房的两位室友。由于没有校内宿舍,我拥有了一段租房上学的新鲜经历。虽说是微信群里开盲盒聚在一起的室友,却开启了一段珍贵的友谊。
    清早一起卖力蹬车进学校开启充满活力的一天,约定的每周聚餐令人心神放松。遇到阳光明媚的假期满城找公园散步,疫情居家期间三人手忙脚乱的做饭让我这个独生子女也体验了一段仿佛有亲兄弟的日子。
    总之三个人和谐的交流沟通除去了许多生活的烦恼,也让我们各自对未来的发展有了更清晰的认识。收获了如此愉快的租房生活,我对二位室友表示真诚的感谢。
    \par
    同时感谢实习期间的同事们。这一段接近半年的实习是我第一次正儿八经的线下实习经历。记得第一次踏入办公楼大门时的紧张与不知所措,在被大家主动约饭和称兄道弟的交谈中渐渐消散。
    游戏引擎的开发是一个有些难度的领域,入职前我也只是有一个模糊的认识。导师给了我充足的时间去做了解,并在恰当的时机给予实际的项目问题,让我在解决问题的过程中对于游戏引擎的整体架构有了进一步的认识,更是增长了自身解决问题的自信心。
    这些可爱可敬的同事让我体验到了劳逸结合的工作氛围,也让我认识到今后的工作生涯需要找到持续学习与享受生活平衡点,才能更好的应对可能出现的风险。这段实习经历让我在知识和心理上都获得了成长。
    \par
    最后感谢人机交互实验室的导师和同学。在徐歆驰学长的带领下,我们小组实现了VR绘制的部分功能,在实验中体会到了VR在游戏之外的重要意义。这也增强了我对于游戏引擎的理解,为我的职业选择指明了道路。邰天成和张蓉蓉两位同学不仅是实验中鼎力相助的好搭档,
    更是成为了生活中分享轶事和烦恼的好伙伴。冯桂焕老师在实验室中不仅给予技术问题上的引导,更是努力让我们探索软件工程的本源,带领我们从软件角度思考社会问题。最后的毕业设计阶段面对形形色色的项目背景也总能迅速提出高屋建瓴的建议。
    遇到如此可爱的实验室大家庭是我的幸运。
    }
\end{acknowledgement}