\chapter{总结与展望}
\section{项目总结}
在国际局势紧张的大背景下,掌握技术主权愈发关键,也是我国正在紧锣密鼓进行中的任务,这两年在各行业也涌现了惊喜的突破。在关系到万家百姓的航空领域,我们已经拿出了C919这一成果,相信不久之后便能迎来国产飞机的第一批旅客。
而在相关的全动飞行模拟机方面仍有许多空白。本文便介绍了一个目标为搭载至D级模拟机上的视景系统的初期工作。以让飞机模型正确飞行为目标,让游戏引擎与仿真机成功交流是本文的主要工作。
\par
为了实现游戏引擎与仿真机的交流,系统中引入了虚拟仿真机这一角色。其负责让讲着数据链路层语言的仿真机与讲着网络层语言的游戏引擎顺利沟通,其核心流程是读取网卡的数据帧,解读其中的指令代码,将指令中的信息重新整合为结构体,最终按规定数据交换协议封装发送。
当然也包括游戏引擎产生反馈数据的逆过程。在实施渲染工程中格外强调效率,因此数据交换协议采用了封装效率较高的ProtoBuffer,同时禁止了有碍于小数据包发送效率的Nagle算法。
此外为了帧率的稳定设计了数据发送时机的限制器,可以按照60Hz的帧率发送数据,避免飞行画面产生顿挫效果。同时为适配不同协议的各色仿真机,模拟仿真机设计了接口层,将不同种类仿真机的信息整合统一为指令结构体,为今后的适配留下空间。
\par
文章行文介绍了视景系统的项目背景介绍了飞行模拟机、视景系统在国内外的技术发展历程,看出我国在该领域有许多空白。
接着一章介绍了系统中涉及到的相关技术和专业概念帮助理解。
在第三章中描述了系统需求分析,使用用例图和用例规格描述了九个用例。给出了数据交换系统的整体架构图,结合核心类图等阐述了每个模块的详细设计。
在第四章的系统实现中。给出了各个模块的核心代码。
最后对于整个系统在PC以及真实FFS上进行部署测试,测试了视景系统的实际运行效果。
\section{项目展望}
虽然该视景系统已经取得一定的效果,但显然距离终极目标仍有距离,需要在后续的工作中安排完成。
\begin{itemize}
    \item [(1)]
    在多种仿真机适配方面虽然在系统结构上留下了空间,但实际操作比较困难。因为难以与仿真机厂商达成合作,只能依赖现有文档去做逆向工程探索仿真机的私有协议结构与内容,这会是极其繁琐的工作。
    期待能够与未来的国产FFS厂商通力合作,不再经历如此费力的工作。
    \item [(2)]
    从整个视景系统来看,距离达到D级模拟机的标准任重道远,作为新产品必将经过严格的审查,各类灯光、天气、突发事件的模拟必须准确到位。实现几百条的细节要求仍需要整个团队相当时间的通力合作。
    \item [(3)]
    此视景系统是基于腾讯自研游戏引擎搭建,这不仅迈出国产视景系统的一步,也是又一自研引擎成长的过程,当前综合渲染效果及稳定性肯定不及成熟的商业游戏引擎。随着该游戏引擎的逐步完善,整个视景系统的体验必然会越来越好。
    
\end{itemize}
